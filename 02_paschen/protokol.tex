\documentclass{protokol}

\usepackage[czech]{babel}
\usepackage[utf8]{inputenc}
\usepackage{icomma}

% Plovouci bloky (obrazky, tabulky)
\usepackage{floatrow}
\floatsetup[table]{capposition=top}
\floatsetup[figure]{frameset={\fboxsep16pt}}
\usepackage{subcaption}

% Tabulky
\usepackage{tabu}
\usepackage{booktabs}
\usepackage{csvsimple}
\usepackage{multirow}
\usepackage{multicol}

% Jednotky
\usepackage{siunitx}
\sisetup{
	locale               = DE,
	inter-unit-product   = \ensuremath{{}\cdot{}},
	list-units           = single,
	list-separator       = {; },
	list-final-separator = \text{ a },
	list-pair-separator  = \text{ a },
	range-phrase         = \text{ až },
	range-units          = single,
}
\usepackage{amsmath}

% Obvody
\usepackage{circuitikz}

% Obrazky a grafy
% \usepackage{graphicx}
\graphicspath{
	{plots/}
	{build/plots/}
}
\usepackage{epstopdf}
\epstopdfsetup{outdir=./build/plots/}

\jmenopraktika={Praktikum z fyziky plazmatu}  % jmeno predmetu
\jmeno={Pavel Kosík, Jan Slaný}                             % jmeno mericiho
\obor={F}                               % zkratka studovaneho oboru
\skupina={Út 15:00}                     % doba vyuky seminarni skupiny
\rocnik={IV}
\semestr={VIII}

\cisloulohy={02}
\jmenoulohy={Paschenův zákon}

\datum={1. března 2022}                  % datum mereni ulohy
\tlak={}% [hPa]
\teplota={}% [C]
\vlhkost={}% [%]

\begin{document}
\header

\section{Teorie}
\subsection{Samostatný výboj a Paschenův zákon}

Elektrický výboj v plynu vzniká tehdy, pokud za působení ionizačního
činidla dodáme plynu dostatečnou energii ke vzniku nabitých částic.
Pokud je pro udržení výboje potřebná přítomnost ionizačního činidla,
jedná se o výboj nesamostatný. V opačném případě nazvěme výboj samostatným.
Hranicí mezi těmito typy výboje je zápalné napětí,
při kterém je vnější elektrické pole schopno v plynu vyvolat lavinovou ionizaci.
Samostatné výboje se dle svých vlastností mohou dále dělit např. na doutnavé, obloukové, jiskrové, koronové.
Změnou podmínek výboje dosahujeme různých zápalných napětí.
Křivka popisující tento jev se nazývá Paschenova křivka.
Paschenův zákon pak dává do souvislosti vzdálenost elektrod $d$ ve výbojovém prostoru
s tlakem plynu $p$ a zápalným napětím $V_z$ podle vzorce:

\begin{equation}
V_z = \frac{Bpd}{C+\ln(pd)}
\end{equation}
kde $B=\frac{U}{\lambda}$ a $C= \ln(\frac{1}{\gamma}+1) + \ln(\frac{1}{\lambda})$.
V tomto zápise $U$ představuje ionizační energii plynu,
$\lambda$ střední volnou dráhu elektronu při jednotkovém tlaku
a $\gamma$ koeficient sekundární emise.

\end{document}
